\documentclass[a4paper,12pt]{article} % добавить leqno в [] для нумерации слева
\usepackage{cmap}					% поиск в PDF
\usepackage{mathtext} 				% русские буквы в фомулах
\usepackage[T2A]{fontenc}			% кодировка
\usepackage[utf8]{inputenc}			% кодировка исходного текста
\usepackage[english,russian]{babel}	% локализация и переносы
\usepackage[left=1cm,right=1cm,
    top=2cm,bottom=2cm,bindingoffset=0cm]{geometry}
    \usepackage[argument]{graphicx}
    \usepackage{graphicx}
    
    \usepackage[table,xcdraw]{xcolor}
    \usepackage{indentfirst}
    
\graphicspath{{./pictures/}}  % папки с картинками
\setlength\fboxsep{3pt} % Отступ рамки \fbox{} от рисунка
\setlength\fboxrule{1pt} % Толщина линий рамки \fbox{}
\usepackage{wrapfig} % Обтекание рисунков и таблиц текстом
    
    

%%% Дополнительная работа с математикой
\usepackage{amsmath,amsfonts,amssymb,amsthm,mathtools} % AMS
\usepackage{icomma} % "Умная" запятая: $0,2$ --- число, $0, 2$ --- перечисление

%% Номера формул
\mathtoolsset{showonlyrefs=true} % Показывать номера только у тех формул, на которые есть \eqref{} в тексте.

%% Шрифты
\usepackage{euscript}	 % Шрифт Евклид
\usepackage{mathrsfs} % Красивый матшрифт

%% Свои команды
\DeclareMathOperator{\sgn}{\mathop{sgn}}

%% Перенос знаков в формулах (по Львовскому)
\newcommand*{\hm}[1]{#1\nobreak\discretionary{}
{\hbox{$\mathsurround=0pt #1$}}{}}

%% ATTENTION 
\usepackage{calc}
\usepackage{wrapfig}
\usepackage{setspace}
\usepackage{indentfirst}
\usepackage{subfigure}
\usepackage[utf8]{inputenc}
\usepackage[russian]{babel}
\usepackage[OT1]{fontenc}
\usepackage{amsmath}
\usepackage{amsfonts}
\usepackage{amssymb}
\usepackage{graphicx}
\graphicspath{{Images/}}
%% ATTENTION 

\begin{document}

\section*{Билет №8}

\subsection*{Геометрические приложения определенного интеграла}

\textbf{Площадь криволинейной трапеции}
\vspace{20}

\textbf{Опрделение:}


Пусть на $[a, b]$ задана функция $f:\; \forall x \in[a, b] \rightarrow f(x) \geq 0$

Множество $G=\{(x,y): a\leq x \leq b, 0 \leq y \leq f(x)\}$ называется криволинейной трапецией

Интегрируемость криволиенйной трапеции по Жордану была доказана ранее

\vspace{20}

\textbf{Предложение:}

Площадь $m(X)$ криволинейной трапеции $X$ определяется формулой $m(X) = \int\limits_a^b f(x)dx$

\textbf{Доказательство:} 

$f$ интегр. на $[a, b] \Rightarrow \; \forall \varepsilon > 0 \; \exists T: \overline{S_T} - \underline{S_T} < \varepsilon$

НО $m(G_{\varepsilon}) = \underline{S_T} \leq I \leq \overline{S_T} = m(G^{\varepsilon}) $

$m(G_{\varepsilon}) \leq m(x) \leq m(G^{\varepsilon})$

$m(x) = I = \int\limits_a^b f(x)dx$

\vspace{20}

\textbf{Площадь криволинейного сектора}

\textbf{Определение:}

$r=r(\varphi)$ непр. на $[\alpha, \beta]$

Криволинейный сектор $Х$ измерим по Жордану

\textbf{Приложение:}

Площадь $m(X)$ криволинейного сектора $Х$ вычисляется по формуле $m(X) =  \frac{1}{2} \int\limits_{\alpha}^{\beta} r^2(\varphi)d\varphi$

\textbf{Доказательство:}

$T = \{\alpha = \varphi_0 < \varphi_1 <\dots < \varphi_n = \beta\}$

$\Delta \varphi_i = \varphi_i - \varphi_{i-1}, i = \overline{1, n}$

$R_i = \max r(\varphi)\; на\; [\varphi_i, \varphi_{i-1}]\;\;\;\;\;\;\;\;\; r_i=\min r(\varphi)\; на\; [\varphi_i, \varphi_{i-1}] $

$\overline{S_T} = \frac{1}{2}\sum\limits_{i=1}^n R_i^2 \Delta \varphi_i,\;\;\;\;\;\;\;\;\;\;\; \underline{S_T} =\frac{1}{2}\sum\limits_{i=1}^n r_i^2 \Delta \varphi_i$

Это верхняя и нижняя сумма Дарбу функции $\frac{1}{2}r^2(\varphi)$

Это функция интегр. на $[a, b]$

$\underline{S_T} \leq I \leq \overline{S_T}$ и $I = m(X) = \frac{1}{2}\int\limits_{\alpha}^{\beta}r^2(\varphi)d\varphi$

\textbf{Объем тела вращения:}

\textbf{Определение:}

Тело, полученное путем вращения криволинейной трапеции вокруг оси $Оx$ наз телом вращения

\vspace{20}

\textbf{Предложение:}

Объем $m(X)$ тела вращения $X$ кривол трапеции вокруг $Оx$ выч. по формуле $m(X) = \pi \int\limits_a^b[f(x)]^2dx$

\textbf{Доказательство:} $T={a=x_0<x_1<\dots<x_n = b}$

$M_i = \max (f(x_{i-1}), f(x_i)), \;\;\;\;\;\;\; m_i=\min (f(x_{i-1}), f(x_i))$

$\overline{S_T} = \pi \sum\limits_{i=1}^n M_i^2 \Delta x_i, \;\;\;\;\;\;\;\;\; \underline{S_T} = \pi \sum\limits_i^n m_i^2 \Delta x_i$

Это суммы Дарбу функции $y = \pi f^2(x)$ кот. интегр. на $[\varepsilon_i, b]$

$\underline{S_T} \leq m(X) = I \leq \overline{S_T} \Rightarrow$ $m(X) = I = \pi \int\limits_a^b f^2(x)dx$

\vspace{20}

\textbf{Длина дуги кривой}

$\Gamma = \{\Vec{r} = \Vec{r}(t)S,\;\;\; \alpha \leq t \leq \beta\}$ непр дифф $\Rightarrow$ спрямляемая кривая

\textbf{Предложение:}

Если кривая $\Gamma$ непр. дифф., то ее длина $L$ вычисляется по формуле $L=\int\limits_a^b |\Vec{r}\,'(t)|dt$

\textbf{Доказательство} $S'(t)=|\Vec{r}\,'(t)|$

$L=S(\beta)-S(\alpha)=\int\limits_\alpha^\beta S'(t)dt = \int\limits_\alpha^\beta |\Vec{r}\,'(t) dt|$

\begin{enumerate}
    \item $\Vec{r}=\Vec{r}(t)=\{ x(t), y(t), z(t) \}, t\in [\alpha, \beta]$
    
    $L = \int\limits_\alpha^\beta \sqrt{[x'(t)]^2+[y'(t)]^2+[z'(t)]^2}dt$
    
    \item $y=f(x)$, $\alpha \leq x \leq \beta$
    
    $\L=\int\limits_\alpha^\beta \sqrt{1+[f'(x)]^2}dx$
    
    \item $r=r(\varphi), \varphi_1 \leq \varphi \leq \varphi_2$
    
    \begin{equation*}
 \begin{cases}
  x=r \cos(\varphi)
   \\
   y = r \sin(\varphi)
   
 \end{cases}
\end{equation*}

 \begin{equation*}
 \begin{cases}
  x'=r' \cos(\varphi) - r\sin(\varphi)
   \\
   y' = r' \sin(\varphi) + r\cos(\varphi)
   
 \end{cases}
\end{equation*}
\begin{equation}
  (x')^2+(y')^2=(r')^2+(r)^2  
\end{equation}

\begin{equation}
  L =\int\limits_\alpha^\beta \sqrt{(r')^2+(r)^2}d\varphi 
\end{equation}

\end{enumerate}

\subsection*{Вычисление площади поверхнности вращения}

Пусть $y=f(x), x \in [a, b]$ и $f(x)$ непрерывна на $[a, b]$

Рассмотрим поверхность $\Pi$ вращения графика функции $f$ вокруг $Оx$

$T = \{ a = x_0 < x_1<\dots<x_n = b \}$



$A_0 (x_0, f(x_0)), A_1 (x_1, f(x_1)) \dots A_n(x_n, f(x_n))$ 

Строим ломанную $A_0, A_1\dots A_n$

При вращении ломанной вокруг оси $Оx$, получаем поверхность $\Pi_T$, составлящую одну из боковых поверхностей усеченных конусов, обозначим эту площадь за $P_T$

$P_T=2\pi \sum\limits_{i=1}^n \frac{f(x_{i-1})+f(x_i)}{2}l_i=\pi \sum\limits_{i=1}^n [f(x_{i-1})+f(x_i)]l_i,$ где $l_i$ - длина звена $A_{i-1}A_i$

\vspace{20}

\textbf{Определение:} Число $P$ называется пределом площади $P_T$ при мелкости разбиения, стремящимся к нулю, если $\forall \varepsilon > 0 \text{ }\exists \delta=\delta(\varepsilon) > 0 : \forall T: \Delta_T<\delta \rightarrow |P_T-P|<\varepsilon$

\vspace{20}

\textbf{Определение:}  Поверхность $\Pi$ называется квадрируемой, если существует предел площадей $P_T$ при мелкости разбиения,  стремещейся к 0. При этом $P$ называется площадью поверхности $\Pi$.

\vspace{20}

\textbf{Предложение:} Если $y=f(x)$ непрерывно дифференцируема на $[a, b]$, $f(x) \geq 0\; \forall x\in[a,b]$, то поверхность вращения $\Pi$ графика $y=f(x)$ вокруг $Оx$, квадрируема и ее площадь вычисляется по форумуле $P=2\pi \int\limits_a^b f(x) \sqrt{1+[f'(x)]^2}dx$

\textbf{Без доказательства}

\subsection*{Криволинейные интегралы первого рода}


\textbf{Определение:} Если существует предел $I$ интегральной суммы $\sigma_T$ при $\Delta S_T \rightarrow 0$, то этот предел называют \textit{криволинейным интегралом первого рода} функции $f$ по кривой $\Gamma$.

\vspace{20}
\textbf{Обозначение:} $I =\int\limits_Г f(x,y)dS$

\textbf{Определение:} Если существует предел $I$ интегральной суммы $\sigma_T^x\; [\sigma_T^y]$  при $\Delta S_T \rightarrow 0$, то этот предел называют криволинейным интегралом второго рода функции $P(Q)$ по кривой $\Gamma$.

\textbf{Обозначение:} $I = \int\limits_\Gamma P(x,y)dx, \int\limits_\Gamma Q(x,y)dy$

\vspace{20}

Сумму $\int\limits_Г P(x,y)dx + Q(x,y) dy$ называют \textit{криволинейный интеграл второго рода}

\vspace{20}

\textbf{Физический смысл крив. инт. 1-го рода} - это масса кривой $\Gamma$, плотность которой задана функцией $\rho = f(x,y)$.

\vspace{20}

\textbf{Физический смысл крив. инт. 2-го рода} - это это работа по перемещению материальной точки вдоль кривой $\Gamma$ под действием силы, имеющей компаненты $u=P(x,y),\; v=Q(x,y)$.  

\vspace{20}

\textbf{Замечание:} Значение криволинейного интеграла 1-го рода не зависит от направления обхода кривой $\Gamma$.

Для криволинейного интеграла 2-го рода изменение направление обхода меняет знак.

\subsection*{Несобственный интеграл}

\textbf{Определение:} 

Пусть $y=f(x)$ интегр на $[a, \xi]$ $\forall \xi:\xi>a $. Символ $\int\limits_a^{\infty} f(x) dx   $ наз. несобств. интегралом функции $y=f(x)$ по промежутку $[a; +\infty]$.

Если существует и конечен предел $\lim\limits_{\xi \rightarrow \infty} I(\xi) = A, \; A\in R$, то несобственный интеграл $I=\int\limits_a^{\infty} f(x) dx$ наз. сходящимся и равен числу $A$.

\textbf{Обозначение:} $\int\limits_a^{\infty} f(x) dx < 0 \equiv$ интеграл сходится 

\textbf{Соглашение:} несобственный интеграл будет записываться как $\int\limits_a^{b} f(x) dx$, где $b = \infty$ или $b$ - вертикальная асимптота $f(x)$. 

\vspace{20}

\textbf{Свойства несоб. инт. и их вычисление:}

\begin{enumerate}
    \item $\int\limits_a^{b} f(x) dx = \int\limits_a^{c} f(x) dx + \int\limits_c^{b} f(x) dx$

$\forall c:\; a<c<b:$

$$\bigl[\int\limits_a^{b} f(x) dx<\infty \bigr] \Leftrightarrow \int\limits_c^{b} f(x) dx < \infty$$

    \item $\bigl[ \bigl[ \int\limits_a^{b} f(x) dx<\infty\bigr] \And \bigl[ \int\limits_a^{b} g(x) dx<\infty \bigr]\bigr] \Rightarrow \int\limits_a^{b} \bigl[\alpha f(x) + \beta g(x)\bigr] dx = \alpha \int\limits_a^{b} f(x) dx + \beta \int\limits_a^{b} g(x) dx$
    
    \item Пусть $y=f(x)$ непр. на $[a, b)$ и $F'(x) = f(x)$ $\forall x\in[a, b)$ и $\lim\limits_{x\rightarrow b-0} F(x) = F(b-0) \in R$.
    
    Тогда:
    
    $$\int\limits_a^{b} f(x) dx = F(b-0) - F(a)\; -\; Формула\; Ньютона-Лейбница$$ 
    
    
    \item  Интегрирование по частям работает так же как и неопределенных интегралах
    
    \item Пусть $y=f(x)$ непр. на $[a, b)$, $x=\varphi(t)$ непр. дифф. и возр на $[\alpha, \beta),\; \varphi(\alpha)=a,\; 
    \lim\limits_{t\rightarrow \beta - 0}\varphi(t) = b$. Тогда:
    
    $$\int\limits_a^{b} f(x) dx = \int\limits_\alpha^{\beta} f(\varphi(t))\varphi ' (t) dt$$
    при условии сходимости хотя бы одного интегралов равенства
    
    \item $$\bigl[ \bigl[ \int\limits_a^{b} f(x) dx<\infty\bigr] \And \bigl[ \int\limits_a^{b} g(x) dx<\infty \bigr] \And \bigl[ f(x) < g(x) \bigr] \bigr] \Rightarrow \int\limits_a^{b} f(x) dx < \int\limits_a^{b} g(x) dx $$
    
\end{enumerate}

\subsection*{Несобственные интегралы от неотр. функций:}

\textbf{Теорема 1}

$\bigl[ \int\limits_a^{b} f(x) dx <\infty \bigr] \Leftrightarrow \bigl[ I(\xi) = \int\limits_a^{\xi} f(x) dx$ \;огр.\; на\; $[a, b)$ $\bigr]$

\textbf{Доказательство:} 

Необходимость: $\int\limits_a^{b} f(x) dx \Rightarrow\; I(\xi)$ ограничена на $[a, b)$.

$\int\limits_a^{b} f(x) dx \stackrel{\text{def}}{=} \exists \lim\limits_{\xi \rightarrow b-0}I(\xi) =A \in R \Rightarrow A = \sup\limits_{\xi \in[a,b)} I(\xi)\text{ }(А$\; - это\; точная\; верхняя\; грань) $ \Rightarrow 0 \leq I(\xi) \leq A \Rightarrow$ функция ограничена.

Достаточность: $I(\xi)$  огр. на $[a, b) \Rightarrow \int\limits_a^{b} f(x) dx < \infty$

$I(\xi)$ \; огр. на $[a, b) \stackrel{\text{def}}{=} \exists C>0 : \forall \xi \in[a, b) \longmapsto 0 \leq I(\xi) \leq C \Rightarrow \exists A = \sup\limits_{\xi \in [a,b)}I(\xi)$

Из $A = \sup\limits_{\xi \in[a,b)} I(\xi)$ следует:

\begin{enumerate}
    \item $\forall \xi \in [a, b) \mapsto I(\xi) \leq A$
    
    \item $\forall \varepsilon > 0\; \exists \xi_\varepsilon \in (a, B) : I(\xi_\varepsilon)>A-\varepsilon$
    
    $\xi_\varepsilon = \delta \; \Rightarrow \; \forall \xi \in(\delta, b) \longmapsto I(\xi) \geq I(\xi_\epsilon)> A-\varepsilon$
\end{enumerate}

Тогда $\forall \varepsilon > 0 \; \exists\delta\in(a, b): \forall \xi \in(\delta, b) \longmapsto 0\leq A-I(\xi) < \varepsilon \; \stackrel{\text{def}}{=}\; \lim\limits_{\xi \rightarrow b-0} I(\xi) = A \in R \eqdef \int\limits_a^{b} f(x) dx < \infty$.

\vspace{20}

\textbf{Теорема 2 (признак сравнения)}

Пусть $\forall x \in [a, b) \longmapsto 0\leq f(x) \leq g(x)$

Тогда:

\begin{enumerate}
    \item $\int\limits_a^{b} g(x) dx < \infty \Rightarrow \int\limits_a^{b} f(x) dx < \infty$
    
    \item $\int\limits_a^{b} f(x) dx = \infty \Rightarrow \int\limits_a^{b} g(x) dx = \infty$
\end{enumerate}

\textbf{Доказательство}

\begin{enumerate}
    \item $\int\limits_a^{b} g(x) dx < \infty \Longleftrightarrow$ (Из\; $Т_1$) $G(\xi) = \int\limits_a^{\xi} g(x) dx \; огр.\; на\; полуинтервале \; \stackrel{\text{def}}{=} \exists C \geq 0: \forall\xi \in [a, b) \longmapsto G(\xi)\leq C$.
    
    $I(\xi) = \int\limits_a^{\xi} f(x) dx \leq \int\limits_a^{\xi} g(x) dx \leq C \; \Rightarrow$ (из \; $Т_1) \int\limits_a^{b} f(x) dx < \infty$.
    
    \item $\int\limits_a^{b} f(x) dx = \infty \Rightarrow \int\limits_a^{b} g(x) dx = \infty$
    
    В противном случае: $\int\limits_a^{b} g(x) dx < \infty \Rightarrow(из\; п_1) \; \int\limits_a^{b} f(x) dx < \infty$
    
    \vspace{20}
    
    \textbf{Следствие (Признак сравнения в предельной форме)}
    
    Если $f(x)>0\; и \; g(x)>0\; \forall x\in[a, b)\; и \; f(x) \thicksim g(x)\; при \; x\rightarrow b-0,\; то \int\limits_a^{b} f(x) dx \; и \; \int\limits_a^{b} g(x) dx$ ведут себя одинаково.
    
    \textbf{Доказательство} 
    
    $\bigl[\lim\limits_{x\rightarrow b-0} \frac{f(x)}{g(x)}=1 \bigr] \Rightarrow \bigl[\varepsilon = 1/2. \;\exists \delta \in(a, b): \forall x\in (\delta,  b) \mapsto |\frac{f(x)}{g(x)}-1|<\frac{1}{2}$
    
    $\frac{1}{2}g(x)< f(x) < \frac{3}{2}g(x)$ Далее просто применяем $Т_2$ и $п_2$.
\end{enumerate}




    


\subsection*{Критерий Коши сходимости несобственных интегралов} 

Пусть функция интегрируема в собст. смысле на промежутке из $[a, b)$

Тогда:

$$\bigl[ \int\limits_a^{b} f(x) dx < \infty \gigr]\; \stackrel{\text{def}}{=} \;  \bigl[\forall \varepsilon > 0\; \exists \delta \in (a, b):\; \forall \xi',\; \xi'' \in (\delta, b) \longmapsto \Bigg|\int\limits_{\xi'}^{\xi''} f(x) dx\Bigg| < \varepsilon \bigr]$$.

\textbf{Доказательство:}

$\bigl[ \int\limits_a^{b} f(x) dx < \infty \bigr] \Longleftrightarrow \bigl[ \exists \lim\limits_{\xi \rightarrow b-0} \int\limits_a^{\xi} f(x) dx = \lim\limits_{\xi \rightarrow b-0} I(\xi) = A \in R \birg] \Longleftrightarrow \bigl[\forall \varepsilon > 0\; \exists \delta \in (a, b):\; \forall \xi',\; \xi'' \in (\delta, b) \longmapsto |I(\xi') - I(\xi'')| < \varepsilon \bigr]$

$|\int\limits_{\xi'}^{\xi''} f(x) dx| < \varepsilon \Longleftrightarrow |\int\limits_{\xi'}^{\xi''} f(x) dx| < \varepsilon$


\subsection*{Абсолютная и условная сходимость несобств. инт.}

\textbf{Определение 1}

Интеграл $\int\limits_{a}^{b} f(x) dx$ наз. абсолютно сход., если $\int\limits_{a}^{b} |f(x)| dx < \infty$

\vspace{20}


\textbf{Предложение}

Если интеграл сходится абсолютно, он сходится условно.

\textbf{Доказательство}

$\forall \xi' \xi'' \in (a, b)$

$|\int\limits_{\xi'}^{\xi''} f(x) dx| \leq \int\limits_{\xi'}^{\xi''} |f(x)| dx$. Тогда по критерию Коши
$|\int\limits_{\xi'}^{\xi''} f(x) dx|$ сходится 
и из $Т_2$ сходится и $\int\limits_{\xi'}^{\xi''} |f(x)| dx$

\vspace{20}

\textbf{Определение 2}

Если $\int\limits_{a}^{b} |f(x)| dx = \infty$, а $\int\limits_{a}^{b} f(x) dx < \infty\;  , то$ $\int\limits_{a}^{b} f(x) dx$ наз. условно сходящимся.

\vspace{20}

\textbf{Предложение}

Если $\int\limits_{a}^{b} g(x) dx$ сходится абсолютно, то $\int\limits_{a}^{b} f(x) dx$ и $\int\limits_{a}^{b} f(x)+g(x) dx$ ведут себя одинаково.

\textbf{Доказательство:} Абсолютная сходимость:

$\int\limits_{a}^{b} |f(x)| dx < \infty \Rightarrow |f(x) +g(x)| \leq |f(x)| + |g(x)| \Rightarrow \int\limits_a^b |f(x) +g(x)| dx < \infty$ 

В другую сторону: $\int\limits_a^b |f(x) +g(x)| dx < \infty \Rightarrow f(x) = \bigl[f(x) + g(x) \bigr]-g(x) \Rightarrow |f(x)| \leq |f(x) + g(x)| \Rightarrow $ по $Т_2\; \int\limits_a^b |f(x)| dx < \infty$ 


\subsection*{Теорема 3 (Признак Дирихле)}

Если выполнены условия:

\begin{enumerate}
    \item $f(x)$ непр, $g(x)$ непр. дифф. на $[a, b)$
    
    \item $F(x)$ = $\int\limits_a^b f(t) dt$ ограничена на $[a, b)$
    
    \item $g(x)$ монотонна на $[a, b)$ и $\lim\limits_{x\rightarrow b-0}g(x) = 0$

\end{enumerate}

Тогда $\int\limits_a^b f(x)\cdot g(x) dx < \infty$

\textbf{Доказательство}:

$\forall \xi', \xi'' \in [a,b)$

Сделаем замену для интегрирования по частям: 

$u = g(x), du=g'(x)dx$

$dv=f(x)dx, v=F(x)$

$\int\limits_{\xi'}^{\xi''} g(x) dx$ = $g(x)\cdot F(x)\bigg|_{\xi'}^{\xi''} - \int\limits_{\xi'}^{\xi''} g'(x)F(x)dx$ 

$|\int\limits_{\xi'}^{\xi''} f(x)g(x)dx|\leq M(|g(\xi')|+|g(\xi'')|) \pm M \int\limits_{\xi'}^{\xi''} g'(x) \leq 2M(|g(\xi')|+|g(\xi'')|)$

$\bigl[ \lim\limits_{x\rightarrow b-0}g(x) = 0 \bigr] \stackrel{\text{def}}{=} \bigl[ \forall \varepsilon > 0 \; \exists \delta(\varepsilon) \in (a, b) \forall x\in (\delta, b) \longmapsto |g(x)| \leq \frac{\varepsilon}{4M} \Rightarrow \forall \; \xi', \xi'' \in (\delta, b) \longmapsto |\int\limits_{\xi'}^{\xi''} f(x)g(x)dx|<2M(\frac{\varepsilon}{4M}+\frac{\varepsilon}{4M}) = \varepsilon \Rightarrow $ по критерию Коши $\int\limits_{a}^{b}f(x)g(x)<\infty$.

\vspace{20}

\textbf{Следствие (Признак Абеля)}

Если выполняются условия:

\begin{enumerate}
    \item $f(x)$ непр, $g(x)$ непр. дифф. на $[a, b)$
    
    \item $\int\limits_{a}^{b}f(x) < \infty$
    
    \item $g(x)$ монотонна и ограничена на $[a, b)$ 

\end{enumerate}

Тогда $\int\limits_a^b f(x)\cdot g(x) dx < \infty$

\textbf{Доказательство}

Из условия 3 следует, что $\exists\lim\limits_{x\rightarrow b-0}g(x) = g(b-0) \in R$

$g_1(x) = g(x) - g(b-0) \xrightarrow[x \to b-0]{}$ 0 $\Rightarrow(по\; Дирихле)$ $\int\limits_a^b f(x)g_1(x) dx < \infty$

$\int\limits_a^b f(x)g_1(x)dx = \int\limits_a^b f(x)g(x)dx+g(b-0)\int\limits_a^b f(x)dx$

$\int\limits_a^b f(x)g_1(x)dx$ и $\int\limits_a^b f(x)dx$ сходятся, значит сходится и $\int\limits_a^b f(x)g(x)dx$










 

\end{document}
