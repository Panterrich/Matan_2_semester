
\documentclass[a4paper,14pt]{extreport}

%%% Работа с русским языком
\usepackage{cmap}					% поиск в PDF
\usepackage{mathtext} 				% русские буквы в формулах
\usepackage[T2A]{fontenc}			% кодировка
\usepackage[utf8]{inputenc}			% кодировка исходного текста
\usepackage[english,russian]{babel}	% локализация и переносы

% Дополнительная работа с математикой
\usepackage{amsmath,amsfonts,amssymb,amsthm,mathtools} % AMS
\usepackage{icomma} % "Умная" запятая: $0,2$ --- число, $0, 2$ --- перечисление

%% Шрифты
\usepackage{euscript}	 % Шрифт Евклид
\usepackage{mathrsfs} % Красивый матшрифт
\usepackage{dsfont}
%% Перенос знаков в формулах (по Львовскому)
\newcommand*{\hm}[1]{#1\nobreak\discretionary{}
	{\hbox{$\mathsurround=0pt #1$}}{}}

%% Русские списки
\usepackage{enumitem}
\makeatletter
\AddEnumerateCounter{\asbuk}{\russian@alph}
\makeatother

%% Поля
\usepackage[left=2cm,right=2cm,top=2cm,bottom=2cm,bindingoffset=0cm]{geometry}

%% Интервалы
\linespread{1}
\usepackage{multirow}

%% TikZ
\usepackage{tikz}
\usetikzlibrary{graphs,graphs.standard}

\usepackage{cancel} % перечеркивания

\begin{document}

\section*{Билет №2.}

\subsection*{Предел числовой функции нелскольких переменных.}

\textbf{Обозначения:}
$\mathscr{M} = (\mathbb{M}, \rho)$, $a \in \mathscr{M}$, $\mathscr{U}(a)$,
$w = f(x)$ - некоторая функция, заданная в $\mathscr{U}$(a), за исключением, быть может, самой точки $a$.
	\\
\textbf{Определение по Гейне:}
	
	$[\lim\limits_{n \to \infty}f(x) = b] \stackrel{def}{=} \left[ \forall \{x^n\}: 
	[x^n \xrightarrow[n \rightarrow \infty]{} a] ~ \& ~ [x^n \neq a ~ \forall n] \mapsto 
	w^n = f(x^n) \xrightarrow[n \rightarrow \infty]{} b\right]$.
\\
\textbf{Определение по Коши:}

	$[\lim\limits_{x \to a}f(x) = b] \stackrel{def}{=} \left[\forall \varepsilon > 0 ~ 
	\exists\delta = \delta(\varepsilon) > 0: 
	[\forall x: ~ 0 < \rho (x, a) < \delta] \mapsto |f(x) - b| < \varepsilon\right]$.
\\	
\textbf{Пример:}
	
	$$w = f(x, y) = \frac{2xy}{x^2 + y^2} ~ \text{,}$$
	$$x^2 + y^2 \neq 0, ~ \vec{0} = (0, 0)$$
	$$[\lim\limits_{(x,y) \to \vec{0}}f(x, y) - \text{не существует}]$$

	Рассмотрим последовательности:

	\

	$\{z^n\}^\text{'} = \{(x^n, y^n)\} = \left(\frac{1}{n}, \frac{1}{n}\right)$ 
	$\quad \rho(\{z^n\}^\text{'}, \vec{0}) = \frac{\sqrt2}{n} \xrightarrow[n \rightarrow \infty]{} 0$

	\
	
	$\{z^n\}^\text{''} = \{(x^n, y^n)\} = \left(-\frac{1}{n}, \frac{1}{n}\right)$
	$\quad \rho(\{z^n\}^\text{''}, \vec{0}) = \frac{\sqrt2}{n} \xrightarrow[n \rightarrow \infty]{} 0$

	\

	Однако
	$f(\{z^n\}^\text{'}) = 1, \quad f(\{z^n\}^\text{''}) = -1$. Поэтому предел функции $f(x, y)$ в точке
	$\vec{0} = (0, 0)$ - не существует.
	\\
\textbf{Предложение:} Пусть $a \in \mathscr{M} \text{ и } w = f(x) \text{, } w = g(x)$ определены
	в $\mathscr{U}(a)$, за исключением, быть может, самой точки $a$;
	$\lim\limits_{x \to a}f(x) = b$, $\lim\limits_{x \to a}g(x) = c$. Тогда:
	\\ 
	$\lim\limits_{x \to a}[f(x) \pm g(x) ] = b \pm c$.
	\\ 
	$\lim\limits_{x \to a}[f(x) \cdot g(x) ] = b \cdot c$.
	\\ 
	$\lim\limits_{x \to a}\left[\frac{f(x)}{g(x)}\right] = \frac{b}{c} \text{, } c \neq 0$.
	
	\

	Доказательство аналогично доказательству для функций одной переменной.
	\\ 
	\textbf{Определение:} Функция $\alpha = \alpha(x) \text{, определенная в }\mathscr{U}(a)$,
	за исключением, быть может, самой точки $a$, называется бесконечно малой, если
	$\lim\limits_{x \to a}\alpha(x) = 0$.
	\\ 
\textbf{Предложение:}
	\\  [2 mm]
	$[f(x): \lim\limits_{x \to a}f(x) = b] \Rightarrow 
	[\alpha = \alpha(x) = f(x) - b -\text{бесконечно малая при } x \rightarrow a]$.

\subsection*{Предел функции по множеству.}
\textbf{Обозначения:}
	$a$ - предельная точка множества $A \subset \mathscr{M}, ~ w = f(x) \text{ определена в } A$.
\\ [2 mm]
\textbf{Определение:} Предел функции по множеству:
\\ [2 mm]
$[\lim\limits_{x \xrightarrow[x \in A]{} a}f(x) = b] \stackrel{def}{=} \left[\forall \varepsilon > 0 ~ 
\exists\delta = \delta(\varepsilon) > 0:
\forall x \in A: ~ 0 < \rho (x, a) < \delta \mapsto |f(x) - b| < \varepsilon\right]$
\\ [2 mm]
\textbf{Обозначения:}
$D \subset \mathds{E}^m$ - неограниченное множество. $w = f(x)$ - определена на $D$.
\\ [2 mm]
\textbf{Определение:} Предел функции при $x \rightarrow +\infty$:
\\ [2 mm]
$[\lim\limits_{x \rightarrow \infty}f(x) = b] \stackrel{def}{=} \left[\forall \varepsilon > 0 ~ 
\exists\delta = \delta(\varepsilon) > 0:
\forall x \in D: \rho (x, \vec{0}) > \delta \mapsto |f(x) - b| < \varepsilon\right]$,
где $\vec{0} = (0, ... , 0)_m$
\\ [2 mm]
\textbf{Определение:} Пусть функция $w = f(x)$ определена на множестве $\prod_r(x_0, y_0) = 
\{(x,y) \in \mathds{E}^2: 0 < |x - x_0| < r_1, 0 < |y - y_0| < r_2\}$
\\ [2 mm]
$\forall x \in (x_0 - r_1, x_0 + r_1), ~ x \neq x_0 ~ 
\exists \lim\limits_{y \to y_0}f(x, y) = \varphi(x), ~ \exists \lim\limits_{x \to x_0}\varphi(x) = b$
\\ [2 mm]
Тогда говорят, что у функции $w = f(x, y)$ существует повторный предел
 $\lim\limits_{x \to x_0} \lim\limits_{y \to y_0}f(x, y) = b$
\\ [5 mm]
 $\forall y \in (y_0 - r_1, y_0 + r_1), ~ y \neq y_0 ~ 
 \exists \lim\limits_{x \to x_0}f(x, y) = \psi(y), ~ \exists \lim\limits_{y \to y_0}\psi(y) = c$
\\ [2 mm]
Тогда говорят, что у функции $w = f(x, y)$ существует повторный предел
  $\lim\limits_{y \to y_0} \lim\limits_{x \to x_0}f(x, y) = c$ 
\\ [2 mm]
\textbf{Замечание:} Из существования предела функции в точке не следует существование повторных пределов. 
А из существования и равенства повторных пределов не следует существования предела в точке.
\\ [2 mm]
\textbf{Примеры:}
\begin{enumerate}
	\item $$w = f(x, y) = \frac{2xy}{x^2 + y^2}, ~ x^2 + y^2 \neq 0$$
	\\$$\lim\limits_{y \to 0} \lim\limits_{x \to 0}f(x, y) = 
	\lim\limits_{x \to 0} \lim\limits_{y \to 0}f(x, y) = 0$$
	Но предел функции в точке $(0, 0)$ не существовует.
	\item $$ w = f(x, y) = 
	\begin{cases}
		x \cdot sin\left(\frac{1}{y}\right), ~ y \neq 0
		\\0, ~ y = 0
	\end{cases}$$
	\\ [2mm]
	$|f(x, y)| \leq |x| \leq \sqrt{x^2 + y^2} < \delta = \varepsilon$ 
	\\ [2mm]
	$\left[\forall \varepsilon > 0 ~ 
	\exists\delta = \delta(\varepsilon) > 0:
	\forall x \in A: ~ 0 < \rho (x, a) < \delta \mapsto |f(x) - b| < \varepsilon\right]$
	\\ [4mm]
	$\lim\limits_{(x, y) \xrightarrow[y \neq 0]{} \vec{0}}f(x, y) = 0 \quad
	\lim\limits_{y \to 0} \lim\limits_{x \to 0}f(x, y) = 0 \text{, однако }
	\\ \lim\limits_{x \to 0} \lim\limits_{y \to 0}f(x, y) - не ~существует.$
\end{enumerate}

\textbf{Предложение:} Пусть $w = f(x, y)$ определена в 	$\prod_r(x_0, y_0) = 
\{(x,y) \in \mathds{E}^2: 0 < |x - x_0| < r_1, ~ 0 < |y - y_0| < r_2\} ~ \text{ и } \lim\limits_{(x, y) \to (x_0, y_0)}f(x, y) = b$. Пусть, кроме того, $\forall x: ~
0 < |x - x_0| < r_1 ~ \exists \lim\limits_{y \to y_0}f(x, y) = \varphi(x)$ и 
$\forall y: ~ 0 < |y - y_0| < r_2 ~ \exists \lim\limits_{x \to x_0}f(x, y) = \psi(y)$.
Тогда повторные пределы существуют и равны числу $b$.


\subsection*{Непрерывность функции нескольких переменных в точке и по множеству.}
\textbf{Определение:} Функция $w = f(x)$, определенная в $\mathscr{U}(a) \subset \mathscr{M}$ называется непрерывной в точке $a$, если $\lim\limits_{x \to a}f(x) = f(a)$.
\\[2mm]\textbf{Обозначения:} $ w = f(x)$ определена на $A \subset \mathscr{M}$ и $a$ предельная точка множества $A$.
\\[2mm]\textbf{Определение:} Функция $w = f(x)$ называется непрерывной в точке $a$ по
множеству $A$, если $\lim\limits_{x \xrightarrow[x \in A]{} a}f(x) = f(a)$.
\\[2mm]\textbf{Определение:} Функция $w = f(x)$ называется непрерывной на множестве 
$\mathbb{X} \subset \mathscr{M}$, если она непрерывна в каждой точке множества 
$\mathbb{X}$ по множеству $\mathbb{X}$.
\\[2mm]\textbf{Предложение:} $[f - \text{непрерывна в точке }a \in \mathscr{M}] \Leftrightarrow [\Delta f(x) = f(x) - f(a) - \text{бесконечно малая при } x \to a]$
\\[2mm]\textbf{Обозначения:} $w = f(x), ~ x \in \mathbb{E}^m; \quad \Delta_kf(x^0, \Delta x_k) = f({x_1}^0, ..., {x_{k-1}}^0, {x_{k}}^0 + \Delta x_k, {x_{k+1}}^0, ..., {x_{m}}^0) - f(x^0)$.
\\[2mm]
Частичное приращение функции $w = f(x)$ в точке $x^0 = ({x_1}^0, ..., {x_m}^0)$ соответствуют приращению $\Delta x_k$ аргумента $x_k$.
\\[2mm]\textbf{Определение:} Функция $w = f(x)$ называется непрерывной в точке $x^0$ по переменной $x_k$, если $\lim\limits_{\Delta x_k \to 0}\Delta_k f(x^0, \Delta x_k) = 0$
\\[2mm]\textbf{Замечание:} Из непрерывности функции $w = f(x)$ в точке $x^0 = ({x_1}^0, ..., {x_m}^0)$ следует непрерывность функции по каждой переменной, но из непрерывности функции по каждой переменной не следует непрерывность функции в точке.
\\[5mm]\textbf{Контрпримеры:}
\begin{enumerate}
	\item $$w = f(x, y) = 
		\begin{cases}
			\frac{xy}{x^2 + y^2}, ~ x^2 + y^2 \neq 0;
			\\0, ~ x^2 + y^2 = 0.
		\end{cases}
		$$
		$$\Delta_xf(\vec{0}, x) = \Delta_yf(\vec{0}, y) = 0.$$
		Функция непрерывна в точке $\vec{0} = (0, 0)$ по переменной $x$ и по переменной $y$.
		Однако пусть $y = kx$, тогда:
		\\[3mm]$\lim\limits_{(x, y) \to \vec{0}}f(x, y) = \lim\limits_{x \to 0}\frac{kx^2}{(1 + k^2)x^2} =
		\frac{k}{1+k^2} \neq 0$, при $k \neq 0$. Поэтому функция $f(x, y)$ не является непрерывной в точке $\vec{0}$.
	\item $$w = f(x, y) = 
		\begin{cases}
			\frac{x^2y}{x^4+y^2}, ~x^2 + y^2 \neq 0,
			\\0, x^2 + y^2 = 0;
		\end{cases}
		$$
		Функция $f$ непрерывна в точке $\vec{0}$ по переменной $x$ и по переменной $y$,
		непрерывна по множеству $y = kx$, однако не является непрерывной в точке $\vec{0}$
		по множеству $y = x^2$:
		$\lim\limits_{(x, y) \to \vec{0}}f(x, y) = \frac{1}{2} \neq 0$.
\end{enumerate}
\subsection*{Свойства функций, непрерывных на компакте: ограниченность, достижение точных нижней и верхней граней, равномерная непрерывность (теорема Кантора).}
\textbf{Предложение:} Пусть функции $w = f(x)$ и $w = g(x)$ непрерывны в точке $a \in \mathscr{M}$. Тогда функции $f \pm g, ~ f\cdot g, ~ \frac{f}{g} \text{ - непрерывны в точке } a$, 
в случае частного $g(a) \neq 0$.
\\[2mm]\textbf{Обозначения:} $x \in \mathbb{E}^m, ~ x_j = \varphi_j(t), ~t \in T \subset \mathbb{E}^k$,
$j = 1, ..., m;~\forall t \in T \subset \mathbb{E}^k \mapsto x \in \mathbb{X} \subset \mathbb{E}^m$.
На $T \subset \mathbb{E}^k$ определена сложная функция $$F(t) = f(\varphi_1(t),~ ..., ~\varphi_m(t))$$
\textbf{Теорема о непрерывности суперпозиции функций:} Пусть функция $x_j = \varphi_j(t),~ j = 1, ..., m$,
непрерывна в точке $b = (b_1, ..., b_m)$, причем $b_j = \varphi_j(a), ~ j = 1, ..., m$.
Тогда функция $F(t) = f(\varphi_1(t), ..., \varphi_m(t))$ непрерывна в точке $a$.
\\[2mm]\textbf{Доказательство:}\\ $[w = f(x) \text{ непрерывна в точке } b] \stackrel{def}{=}$ $[\forall \varepsilon > 0~\exists\delta = \delta(\varepsilon) > 0:\\$
$[\forall x: ~  \rho (x, a) < \delta] \mapsto |f(x) - f(b)| < \varepsilon]$.
\\$[\varphi_j \text{ непрерывна в точке } a, ~ j = 1, ..., m] \stackrel{def}{=}$ $[\forall \delta > 0~\exists\sigma_j = \sigma_j(\varepsilon) > 0,$\\[2mm] $j = 1, ..., m:$
$[\forall t: ~ \rho (t, a) < \sigma_j] \mapsto |\varphi_j(t) - \varphi_j(a)| < \frac{\delta}{\sqrt{m}}]$.
$\\[2mm]\exists \sigma = \sigma(\varepsilon) = min\{\sigma_1, ..., \sigma_m\} \Rightarrow \forall t: \rho(t, a) < \delta \Rightarrow |x_j - b_j| < \frac{\delta}{\sqrt{m}}$.
\\[2mm]$\rho(x, b) = \left[\sum\limits_{j = 1}^{m}(x_j - b_j)^2\right]^{1/2} < \left[\sum\limits_{j = 1}^{m}\frac{\delta^2}{m}\right]^{1/2} = \delta \mapsto |f(x) - f(a)| < \varepsilon \Rightarrow$
\\[2mm]$|f(\varphi_1(t), ..., \varphi_m(t)) - f(\varphi_1(a), ..., \varphi_m(a))| < \varepsilon \Rightarrow |F(t) - F(a)| < \varepsilon \Rightarrow$ 
\\[4mm]$F(t)$ - непрерывна в точке $a$ по определению.
\\[4mm]\textbf{Теорема о локальном сохранении знака непрерывной функции:} пусть $w = f(x)$ определена на $\mathscr{U}(a) \subset \mathbb{E}^m$ и непрерывна в точке $x = a,~ f(a) \neq 0.$
Тогда $\exists \delta > 0: \forall x: \rho(x,~ a) < \delta \mapsto f(x)\cdot f(a) > 0.$
\\[2mm]\textbf{Доказательство:} используется "$\varepsilon$ - $\delta"\quad$определение непрерывности функции
функции в точке и выбором $0 < \varepsilon < |f(a)|.$
\\[2mm]\textbf{Tеорема Вейерштрасса:} Пусть функция $w = f(x)$ непрерывна на
компакте $\mathbb{E} \subset \mathbb{R}^n.$ Тогда она ограничена на $\mathbb{E}$ и достигает на $\mathbb{E}$
своих верхней и нижней граней.
\\[2mm]\textbf{Доказательство(по Бесову):} проведем доказательство лишь для случая верхней грани. 
Как увидим, оно повторяет доказательство теоремы Вейерштрасса для случая $n = 1, ~ \mathbb{E} = [a, ~b]$.
\\[2mm]Пусть $B := {\underset{\mathbb{E}}{sup}} f \leq +\infty$. Из определения верхней грани следует, что существует
последовательность точек $\{x^{(m)}\}, ~x^{(m)} \in \mathbb{E} ~\forall m \in \mathbb{N}$ такая, что 
\\[2mm]$\lim\limits_{m \to \infty}f(x^{(m)}) = B$. Последовательность $\{x^{(m)}\}$ ограничена в силу ограниченности
множества $\mathbb{E}$. В силу теоремы Больцано-Вейерштрасса выделим из $\{x^{(m)}\}$ сходящуюся подпоследовательность 
${\{x^{(m_k)}\}}_{k = 1}^\infty$. Пусть $x^{(0)} = \lim\limits_{k \to \infty} x^{(m_k)}$. Точка $x^{(0)}$ принадлежит $\mathbb{E}$
в силу замкнутости $\mathbb{E}$. Следовательно, $f$ непрерывна в точке $x^{(0)}$ по множеству $\mathbb{E}$.

\

Теперь из соотношений 
$$f(x^{(m_k)}) \to B, ~ f(x^{(m_k)}) \to f(x^{(0)}) \text{ при } k \to \infty$$ вытекает, что $f(x^{(0)}) = B$, т.е. что верхняя
грань функции $f$ достигается в точке $x^{(0)} \in \mathbb{E}$, Следовательно, верхняя грань $\underset{\mathbb{E}}{sup}f$ конечна,
а функция $f$ ограничена сверху на $\mathbb{E}$.

\

Аналогично доказывается, что функция $f$ достигает своей нижней грани на $\mathbb{E}$ и ограничена снизу на $\mathbb{E}$. Теорема доказана. 
\\[2mm]\textbf{Определение:} функция $f$ называется равномерно непрерывной на множестве $\mathbb{X} \subset \mathbb{R}^n$, 
если для любого положительного числа $\varepsilon$ найдется положительное число $\delta$ такое, что
для всех точек $x', ~ x" \in \mathbb{X}$, таких, что $\rho(x', ~x") < \delta$, выполняется неравенство
$|f(x') - f(x")| < \varepsilon$. \\На языке кватноров:
\\[2mm]$\forall \varepsilon > 0 ~\exists\delta > 0:$ $\forall x', ~x" \in \mathbb{X}, ~\rho(x', x") < \delta \mapsto |f(x') - f(x")| < \varepsilon$
\\[2mm]\textbf{Теорема Кантора:} Пусть функция $f$ непрерывна на компакте $\mathbb{E} \subset \mathbb{R}^n$. Тогда $f$
равномерно непрерывна на $\mathbb{E}$.
\\[2mm]\textbf{Доказательство(по Бесову):} Предположим, что теорема неверна, то есть, что существует $f$, непрерывная, но не равномерно
непрерывная на $\mathbb{E}$. Тогда:
\\$ \exists \varepsilon_0 > 0 : \forall \delta > 0 ~\exists x, ~y \in \mathbb{E}: |x - y| < \delta: ~ |f(x) - f(y)| \geq \varepsilon_0$
\\Будем в качестве $\delta$ брать $\delta_m = \frac{1}{m}$ и обозначать через $x^{(m)}, ~y^{(m)}$ соответствующую пару точек $x, ~y$. Тогда имеем:
$$x^{(m)}, ~y^{(m)} \in \mathbb{E}, ~|x^{(m)} - y^{(m)}| < \frac{1}{m},$$
$$|f(x^{(m)}) - f(y^{(m)})| \geq \varepsilon_0 > 0.$$
Выделим из последовательности ${x^{(m)}}$ сходящуюся подпоследовательность $\{x^{(m_k)}\}_{k = 1}^\infty$,
$\lim\limits_{k \to \infty}x^{(m_k)} = x^{(0)}$, что возможно по теореме Больцано-Вейерштрасса в силу ограниченности
$x^{(m)}$. Тогда из $|x^{(m)} - y^{(m)}| < \frac{1}{m}$ следует, что $\lim\limits_{k \to \infty}y^{(m_k)} = x^{(0)}$.
Точка $x^{(0)} \in \mathbb{E}$, так как $\mathbb{E}$ замкнуто. В силу непрерывности $f$ в точке
$x^{(0)}$ по множеству $\mathbb{E}$ имеем:
$\quad|f(x^{(m_k)}) \to f(x^{(0)})|$, $\quad|f(y^{(m_k)}) \to f(x^{(0)})|$, \\ при $k \to \infty$, так что
$$|f(x^{(m_k)}) - f(y^{(m_k)})| \leq  |f(x^{(m_k)}) - f(y^{(0)})| + |f(y^{(m_k)}) - f(x^{(0)})| \to 0 \text{, при } k \to \infty$$
Это противоречит тому, что 
$$|f(x^{(m_k)}) - f(y^{(m_k)})| \geq \varepsilon_0 > 0 ~\forall k \in \mathbb{N}$$
Теорема доказана.
\subsection*{Теорема о промежуточных значениях функции, непрерывной в области.}
\textbf{Теорема о прохождении непрерывной функции через любое промежуточное значение:} Пусть функция $w = f(x)$ непрерывна на линейно связном множестве $\mathbb{X} \subset \mathbb{E}^m, ~ a, ~ b \in \mathbb{X} ~ и ~ f(a) = A, ~ f(b) = B$. Пусть число $C$ лежит между числами $A$ и $B$. Тогда на любой кривой $Г$ соединяющей точки $a$ и $b$ и лежащей в $\mathbb{X}$, найдется точка $c$, такая, что $f(c) = C$.
\\[2mm]\textbf{Доказательство:} Пусть $[\alpha, \beta] \subset \mathbb{E}^1$, $x_j = \varphi_j(t)$, $\varphi_j(\alpha) = a_j$, $\varphi_j(\beta) = b_j$, $j = 1, ..., m$;
$\quad a = (a_1, ..., a_m)$, $\quad b = (b_1, ..., b_m)$, $\quad \varphi_j$ непрерывна на $[\alpha, \beta]$.
$Г = \left\{\varphi_1(t),~ ..., ~\varphi_m(t), ~\alpha \leq t \leq \beta \right\}$ соединяющая точки $a$ и $b$, $Г \subset \mathbb{X}$.
Рассмотрим функцию одной переменной $F(t) = f(\varphi_1(t),~ ..., ~\varphi_m(t))$. По теореме о непрерывности суперпозиции функций $F(t)$ - непрерывна на $[\alpha, \beta]$
$F(\alpha) = A, ~ F(\beta) = B \Rightarrow \exists \gamma \in (\alpha, \beta): ~ F(\gamma) = C$ (т. Больцано - Коши). 
Тогда $c = (\varphi_1(\gamma),~ ..., ~\varphi_m(\gamma)) \Rightarrow f(c) = C.$
\end{document}
