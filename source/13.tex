\input{packages_13}
\usepackage{upgreek}


\newcommand{\eqdef}{\stackrel{\mathrm{def}}{=}}
\newcommand{\ryad}{\sum^{\infty}_{k = 0}}

\begin{document}



\section*{Билет 13.}
\subsection*{Разложение в ряд Тейлора основных элементарных функций: $e^x$, $\cos x$, $\sin x$, $\ln (1 + x)$, $(1 + x)^{\alpha}$ }


\subsubsection*{1. Показательная и гиперболические функции.}



\begin{center}
$y = e^x, \; x \in \mathbb{R}$
\vspace{8pt}

$x \in (-\rho, \, \rho), \; \rho > 0$
\end{center}

Поскольку $(e^x)^{(k)} = e^x $, то $0 < f(x) < e^{\rho}$ и $0 < f(x)^{(k)} < e^{\rho}$. Ряд Тейлора функции $y = e^x$ сходится к ней на $( - \rho, \, \rho)$ по теореме о достаточном условии представимости функции её рядом Тейлора.

$\forall \rho > 0 \Rightarrow R = + \infty$

\[ e^x = \sum^{+\infty}_{k = 0} \frac{x^k}{k!}\]

\[y = \sh x, \; y = \ch x, \; x \in \mathbb{R} \]

\[ \sh x = \frac{e^x - e^{-x}}{2}, \; \ch x = \frac{e^x + e^{-x}}{2} \]

\[ \sh x = \ryad \frac{x^{2k + 1}}{(2k + 1)!} \,, \; \ch x = \ryad \frac{x^{2k}}{(2k)!}, \; R = +\infty\]


\subsubsection*{2. Тригонометрические фунции.}

\begin{center}
$y = \sin x, \; y = \cos x, \; x \in \mathbb{R}$
\vspace{8pt}

$| f^{(k)} (x) | \le 1$, $\forall k = 0, \, 1, \, 2, \, \ldots$
\end{center}
 
$$ 
\sin x = \ryad \frac{(-1)^k x^{2k + 1}}{(2k + 1)!}, \; \cos x = \ryad \frac{(-1)^k x^{2k}}{(2k)!}, \;R = + \infty
$$


\subsubsection*{3. Степенная функция.}

$$y = (1 + x)^{\alpha}, \; \alpha \in \mathbb{R}$$


1) $\alpha = 0, \; y = 1$

2) $\alpha = n, \; n \in \mathbb{N}, \; f(x) = \ryad C^k_n x^k$ - бином Ньютона

3) $\alpha$ - произвольное, $\alpha \in \mathbb{R}$

$$ 
	f^{(n + 1)}(x) = \alpha (\alpha - 1) \ldots (\alpha - n) (1 + x)^{\alpha - (n + 1)} $$

$$
\mathrm{r}_n(x) = \frac{\alpha(\alpha - 1) \ldots (\alpha - n)}{n!} \int_0^x \left(\frac{x - t}{1 + t} \right)^n (1 + t)^{\alpha - 1} dt
$$

Пусть $t = x \tau$, $0 \leqslant \tau \leqslant 1$, тогда $dt = x d \tau$

$$
\mathrm{r}_n(x) =  \frac{\alpha(\alpha - 1) \ldots (\alpha - n)}{n!} x^{n + 1} \int_0^1 \left( \frac{1 - \tau}{1 + x\tau} \right)^n ( 1 + x \tau)^{\alpha - 1} d \tau
$$

Пусть $|x| < 1$, тогда $|1 + \tau x| \geqslant 1 - \tau$



\begin{equation*}
(1+x \tau)^{\alpha-1} \leqslant \beta(x)=
 \begin{cases}
   (1+|x|)^{\alpha-1}, & \alpha \geq 1 \\
   (1-|x|), & \alpha<1
 \end{cases}
\end{equation*}

$ | \alpha | \leqslant m $, $m \in \mathbb{N}$. Тогда $\forall n > m$

\begin{multline*}
\left| \frac{\alpha(\alpha - 1) \ldots (\alpha - n)}{n!} \right| \leqslant \frac{m (m + 1) \ldots (m + n}{n!} \leqslant \frac{(m + n)!}{n!} = \\
= (n + 1)(n + 2) \ldots (n + m)\leqslant (2n)^m
\end{multline*}

В итоге

$$ 
\left| \mathrm{r}_n (x) \right| \leqslant 2^m \beta (x) |x| \cfrac{n^m}{\left(\cfrac{1}{|x|}\right)^n} \xrightarrow{n \rightarrow + \infty} 0
$$

Так как 
$$
a = \frac{1}{|x|} > 1 \hspace{0.5cm} \lim_{n \rightarrow + \infty} \frac{n^m}{a^n} = 0$$

Следовательно 

$$
(1 + x)^{\alpha} = \ryad C_{\alpha}^k x^k, \; C_{\alpha}^k = \frac{\alpha (\alpha - 1) \ldots (\alpha - (k -1))}{k!}, \; |x| < 1, \; R = 1 
$$

\underline{В частности:}

$$
\frac{1}{1 - x} = \ryad x^k, \; \frac{1}{1 + x} = \ryad (-1)^k x^k, \; |x| < 1
$$

\subsubsection*{4. Логарифмические функции.}



$$
y = \ln(1 - x), \; y' = - \frac{1}{1 - x} = - \ryad x^k
$$

$$
y = \ln(1 + x), \; y' = - \frac{1}{1 + x} = \ryad (-1)^k x^k
$$

Раскладываем в интервалах сходимости каждую функцию в ряд Тейлора, а потом почленно интегрируем, и помним, что при почленном интегрировании радиус сходимости не меняется. 

$$ 
y = \ln(1 - x) = - \ryad \frac{x^{k + 1}}{k + 1} = - \sum^{\infty}_{k = 1} \frac{x^k}{k}, \; |x| < 1
$$

$$
y = \ln(1 + x) = \ryad \frac{(-1)^k x^{k + 1}}{k + 1} = - \sum^{\infty}_{k = 1} \frac{(-1)^{k - 1} x^k}{k}, \; |x| < 1
$$
\subsubsection*{5. Обратные тригонометрические функции.}

Обратные тригонометрические функции можно разложить в ряд Тейлора, сначала продифференцировав и воспользовавшись известными результатами.

\subsection*{Разложение в степенной ряд комплекснозначной функции $e^z$}

\textbf{Докажем, что} 

$$
e^z = \ryad \frac{z^k}{k!}, \; R = + \infty
$$

$$
\cos z = \ryad \frac{(-1)^k z^{2k}}{(2k)!}, \;\; \sin z = \ryad \frac{(-1)^k z^{2k + 1}}{(2k + 1)!}, \;\;\; R = + \infty
$$

\textbf{Доказательство:}

Так как $z = x + iy$ и по формуле Эйлера: $e^{i \varphi} = \cos(\varphi) + i \sin(\varphi)$, то 

$$
e^z = e^{x + iy} = e^x \left( \cos y + i \sin y \right)
$$

$$
e^x = \ryad \frac{x^k}{k!}, \; \cos y = \ryad (-1)^k \frac{y^{2k}}{(2k)!}, \; \sin y = \ryad (-1)^k \frac{y^{2k + 1}}{(2k + 1)!}
$$

\begin{multline*}
e^{iy} = cos y + i \sin y = \ryad (-1)^k \frac{y^{2k}}{(2k)!} + i \ryad (-1)^k \frac{y^{2k + 1}}{(2k + 1)!} =  \\ = \ryad \frac{(iy)^{2k}}{(2k)!} + \ryad \frac{(iy)^{2k + 1}}{(2k + 1)!} = \ryad \frac{(iy)^{k}}{k!} = e^{iy}
\end{multline*}

$$
e^z = \ryad \frac{x^k}{k!} \cdot \ryad \frac{(iy)^{k}}{k!}
$$

Докажем, что

$$
\ryad \frac{(z_1 + z_2)^k}{k!} = \ryad \frac{z_1^k}{k!} \cdot \ryad \frac{z_2^k}{k!}
$$

\begin{multline*}
\ryad \frac{(z_1 + z_2)^k}{k!} = \ryad \frac{1}{k!} \cdot \sum_{j = 0}^k C_k^j z_1^j z_2^{k - j} = \ryad  \sum_{j = 0}^k \frac{1}{k!} \cdot \frac{k!}{j! \cdot (k - j)!} z_1^j z_2^{k - j} = \\ = \ryad  \sum_{j = 0}^k \frac{z_1^j}{j!} \cdot \frac{z_2^{k - j}}{(k - j)!} = \frac{z_1^0}{0!} \cdot \frac{z_2^0}{0!} + \left(\frac{z_1^0}{0!} \cdot \frac{z_2^1}{1!} + \frac{z_1^1}{1!} \cdot \frac{z_2^0}{0!} \right) + \\ + \left( \frac{z_1^0}{0!} \cdot \frac{z_2^2}{2!} + \frac{z_1^1}{1!} \cdot \frac{z_2^1}{1!} + \frac{z_1^2}{2!} \cdot \frac{z_2^0}{0!} \right) + \ldots
\end{multline*}

Это можно проиллюстрировать следующим образом:

\begin{table}[h!]
\centering
\resizebox{0.6\textwidth}{!}{%
\begin{tabular}{|c|c|c|c|c|c|}
\hline
     & $u_0$      & $u_1$      & $u_2$      & $u_3$ & ... \\ \hline
$v_0$ & $u_0 \cdot v_0$ & $u_1 \cdot v_0$ & $u_2 \cdot v_0$ & ...  &     \\ \hline
$v_1$ & $u_0 \cdot v_1$ & $u_1 \cdot v_0$ & ...       &      &     \\ \hline
$v_2$ & $u_0 \cdot v_2$ & ...       &           &      &     \\ \hline
$v_3$ & ...       &           &           &      &     \\ \hline
...  &           &           &           &      &     \\ \hline
\end{tabular}%
}
\end{table}

Мы обходим таблицу по диагоналям, так что сумма индексов элементов была константа для каждой группы слагаемых. Тогда действительно:

$$
\ryad \frac{(z_1 + z_2)^k}{k!} = \ryad  \sum_{j = 0}^k \frac{z_1^j}{j!} \cdot \frac{z_2^{k - j}}{(k - j)!} = \ryad \frac{z_1^k}{k!} \cdot \ryad \frac{z_2^k}{k!}
$$

Тогда по доказанной выше лемме:

$$
e^z = \ryad \frac{x^k}{k!} \cdot \ryad \frac{(iy)^k}{k!} = \ryad \frac{(x + iy)^k}{k!} = \ryad \frac{z^k}{k!}
$$

Теперь 

\begin{multline*}
e^{iz} = \ryad \frac{(iz)^k}{k!} = \ryad \frac{(iz)^{2k}}{(2k)!} + \ryad \frac{(iz)^{2k + 1}}{(2k + 1)!} = \\ = \ryad (-1)^k \frac{z^{2k}}{(2k)!} + i \ryad (-1)^k \frac{z^{2k + 1}}{(2k + 1)!}
\end{multline*}


\begin{multline*}
e^{-iz} = \ryad \frac{(-iz)^k}{k!} = \ryad \frac{(-iz)^{2k}}{(2k)!} + \ryad \frac{(-iz)^{2k + 1}}{(2k + 1)!} = \\ = \ryad (-1)^k \frac{z^{2k}}{(2k)!} - i \ryad (-1)^k \frac{z^{2k + 1}}{(2k + 1)!}
\end{multline*}

$$
\frac{e^{iz} + e^{-iz}}{2} = \cos z = \ryad (-1)^k \frac{z^{2k}}{(2k)!}
$$


$$
\frac{e^{iz} - e^{-iz}}{2} = \sin z = \ryad (-1)^k \frac{z^{2k + 1}}{(2k + 1)!}
$$

\textbf{Что и требовалось доказать.}






\end{document}